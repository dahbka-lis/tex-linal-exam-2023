\task{5}
\begin{condition}
    Вставьте вместо звёздочки, ромбика и кружочка подходящие числа таким образом, чтобы линейный оператор $\varphi \colon \R^3 \to \R^3$, имеющий в стандартном базисе матрицу:
    \[
        \begin{pmatrix}
            2/3      & *    & 2/3   \\
            -1/3     & -2/3 & 2/3   \\
            \diamond & 2/3  & \circ
        \end{pmatrix},
    \]
    был ортогональным. Найдите ортонормированный базис, в котором матрица оператора $\varphi$ имеет канонический вид и выпишите эту матрицу. Укажите ось и угол поворота, определяемого оператором $\varphi$.
\end{condition}

Для начала разберёмся с неизвестными значениями. Знаем, что матрица ортогонального оператора является \textit{ортогональной} (столбцы матрицы образуют ортонормированную систему).
\[
    A := A(\varphi, \e)
    =
    \begin{pmatrix}
        2/3      & *    & 2/3   \\
        -1/3     & -2/3 & 2/3   \\
        \diamond & 2/3  & \circ
    \end{pmatrix}
\]

Найдём значения, при которых длины столбцов матрицы $A$ равны 1:
\begin{itemize}
    \item $(A^{(1)}, A^{(1)}) = \d{4}{9} + \d{1}{9} + \diamond^2 = 1 \implies \diamond = \pm \d{2}{3}$

    \item $(A^{(2)}, A^{(2)}) = *^2 + \d{4}{9} + \d{4}{9} = 1 \implies * = \pm \d{1}{3}$

    \item $(A^{(3)}, A^{(3)}) = \d{4}{9} + \d{4}{9} + \circ^2 = 1 \implies \circ = \pm \d{1}{3}$
\end{itemize}

Система векторов называется ортогональной, если попарное скалярное произведение векторов из системы равно нулю. Методом перебора понимаем, что нам подходят такие значения:
\[
    \diamond = -\d{2}{3},
    \quad
    * = \d{1}{3},
    \quad
    \circ = \d{1}{3}
\]

Тогда, перед нами матрица ортогонального оператора $\varphi$:
\[
    A =
    \begin{pmatrix}
        2/3  & 1/3  & 2/3 \\
        -1/3 & -2/3 & 2/3 \\
        -2/3 & 2/3  & 1/3
    \end{pmatrix}
\]

\begin{theorem}
    Классификация ортогональных операторов в трёхмерном евклидовом пространстве:
    \[
        \exists \; \e \text{ - ОНБ } \colon
        A(\varphi, \e) =
        \begin{pmatrix}
            \cos{\alpha} & -\sin{\alpha} & 0    \\
            \sin{\alpha} & \cos{\alpha}  & 0    \\
            0            & 0             & \pm1
        \end{pmatrix}
    \]
\end{theorem}

\textit{Примечательный факт}: $\Spec \varphi$, где $\varphi$ - ортогональный линейный оператор, всегда хранит в себе значение 1 или -1.

\newpage

Посчитаем характеристический многочлен:
\[
    \chi_t(\varphi) = (-1)^3 \det(A - tE) = \d{3t^3 - t^2 - t + 3}{3} = \d{1}{3}(t + 1)(3t^2 - 4t + 3) \implies -1 \in \Spec \varphi
\]

Найдём базис для собственного подпространства $V_{-1}(\varphi)$ через ФСР:
\[
    (A - (-1) \cdot E) = (A + E) =
    \begin{pmatrix}
        5/3  & 1/3 & 2/3 \\
        -1/3 & 1/3 & 2/3 \\
        -2/3 & 2/3 & 4/3
    \end{pmatrix}
    \leadsto
    \begin{pmatrix}
        1 & 0 & 0 \\
        0 & 1 & 2 \\
        0 & 0 & 0
    \end{pmatrix}
    \implies e_3 =
    \begin{pmatrix}
        0  \\
        -2 \\
        1
    \end{pmatrix}
\]

Вектор $e_3$ находится в искомом базисе. Оставшиеся векторы $e_1, e_2$ найдём как $\langle e_3 \rangle^\bot = \langle e_1, e_2 \rangle$ через ФСР:
\[
    (e_3) =
    \begin{pmatrix}
        0 & -2 & 1
    \end{pmatrix}
    \leadsto
    e_1 =
    \begin{pmatrix}
        1 \\
        0 \\
        0
    \end{pmatrix}, \;
    e_2 =
    \begin{pmatrix}
        0   \\
        1/2 \\
        1
    \end{pmatrix}
\]

Заметим, что $e_1, e_2$ - ортогональны друг другу (в ином случае потребовалось бы ортогонализовать), тогда система $(e_1, e_2, e_3)$ является ортогональной. Нормируем:
\begin{itemize}
    \item $f_1 = \d{1}{|e_1|} e_1 = e_1 = (1, 0, 0)$

    \item $f_2 = \d{1}{|e_2|} e_2 = \d{2\sqrt{5}}{5} e_2 =
              \l(0,\; \d{\sqrt{5}}{5},\; \d{2 \sqrt{5}}{5}\r)$

    \item $f_3 = \d{1}{|e_3|} e_3 =\d{\sqrt{5}}{5} e_3 =
              \l(0,\; -\d{2 \sqrt{5}}{5},\; \d{\sqrt{5}}{5}\r)$
\end{itemize}

Получаем систему $\f = (f_1, f_2, f_3)$ - ортонормированный базис, в котором матрица ортогонального оператора имеет канонический вид.

\begin{theorem}
    \[
        \cos{\alpha} = (\varphi(f_1), f_1),
        \quad
        \sin{\alpha} = (\varphi(f_1), f_2)
    \]
\end{theorem}

Найдём $\varphi(f_1)$:
\[
    \varphi(f_1) = A f_1 =
    \begin{pmatrix}
        2/3  & 1/3  & 2/3 \\
        -1/3 & -2/3 & 2/3 \\
        -2/3 & 2/3  & 1/3
    \end{pmatrix}
    \begin{pmatrix}
        1 \\
        0 \\
        0
    \end{pmatrix}
    =
    \begin{pmatrix}
        2/3  \\
        -1/3 \\
        -2/3
    \end{pmatrix}
\]
Теперь вычислим значения $\cos{\alpha}$ и $\sin{\alpha}$:
\begin{itemize}
    \item $\cos{\alpha} = (\varphi(f_1), f_1) = \frac{2}{3}$.

    \item $\sin{\alpha} = (\varphi(f_1), f_2) = -\d{\sqrt{5}}{15} - \d{4 \sqrt{5}}{15} = -\d{\sqrt{5}}{3}$
\end{itemize}

\newpage

Итого, в базисе $\f$ матрица оператора $\varphi$ имеет такой канонический вид:
\[
    A(\varphi, \f) =
    \begin{pmatrix}
        2/3         & \sqrt{5}/3 & 0  \\
        -\sqrt{5}/3 & 2/3        & 0  \\
        0           & 0          & -1
    \end{pmatrix}
\]

Данная матрица является матрицей поворота вокруг вектора $f_3$ на угол $\arccos{\frac{2}{3}}$ с зеркальным отражением.

\answer{5}
\begin{itemize}
    \item Базис
          $
              \f =
              (
              \begin{pmatrix}
                  1 \\
                  0 \\
                  0
              \end{pmatrix}, \;
              \begin{pmatrix}
                  0               \\
                  \d{\sqrt{5}}{5} \\
                  \d{2 \sqrt{5}}{5}
              \end{pmatrix}, \;
              \begin{pmatrix}
                  0                  \\
                  -\d{2 \sqrt{5}}{5} \\
                  \d{\sqrt{5}}{5}
              \end{pmatrix}
              )
          $.

    \item Канонический вид:
          $
              A(\varphi, \f)
              =
              \begin{pmatrix}
                  2/3         & \sqrt{5}/3 & 0  \\
                  -\sqrt{5}/3 & 2/3        & 0  \\
                  0           & 0          & -1
              \end{pmatrix}
          $.

    \item Поворот вокруг вектора $\l(0,\; -\d{2 \sqrt{5}}{5},\; \d{\sqrt{5}}{5}\r)$ на угол $\arccos{\d{2}{3}}$ и зеркальное отражение.
\end{itemize}

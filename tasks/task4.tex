\task{4}
\begin{condition}
    Приведите пример недиагонализуемого линейного оператора $\varphi$ в $\R^2$, для которого оператор
    \mbox{$\varphi^2 - 5 \varphi$} диагонализуем.
\end{condition}

Зададим $\varphi \colon \R^2 \to \R^2$ - линейный оператор с матрицей $A := A(\varphi, \e)$ в некотором базисе $\e$. $\varphi$ недиагонализуем, тогда пусть матрица $A$ имеет жорданову нормальную форму:
\[
    A =
    \begin{pmatrix}
        x & 1 \\
        0 & x
    \end{pmatrix},
    \quad
    x \in \Spec \varphi
\]

\begin{theorem}
    Пусть $V$ - векторное пространство над полем $F$. Тогда $\forall \; \varphi, \psi \in L(V), \;\lambda \in F$ в некотором базисе $\e$ имеем:
    \[
        A(\varphi \circ \psi, \e) = A(\varphi, \e) \cdot A(\psi, \e),
        \quad
        A(\lambda \varphi, \e) = \lambda \cdot A(\varphi, \e)
    \]
\end{theorem}

Вычислим явно, чему будет равна матрица оператора $\varphi^2 - 5 \varphi$:
\[
    A(\varphi^2 - 5 \varphi, \e) = A^2 - 5A =
    \begin{pmatrix}
        x^2 & 2x  \\
        0   & x^2
    \end{pmatrix}
    -
    \begin{pmatrix}
        5x & 5  \\
        0  & 5x
    \end{pmatrix}
    =
    \begin{pmatrix}
        x(x - 5) & 2x - 5   \\
        0        & x(x - 5)
    \end{pmatrix}
\]

Заметим, что $\varphi^2 - 5 \varphi$ диагонализуем, если $2x - 5 = 0$ (так как матрица уже будет иметь диагональный вид). Тогда нам подойдёт значение $x = \d{5}{2}$:
\[
    A(\varphi^2 - 5 \varphi, \e) =
    \begin{pmatrix}
        x(x - 5) & 2x - 5   \\
        0        & x(x - 5)
    \end{pmatrix}
    =
    \begin{pmatrix}
        -\d{25}{4} & 0          \\
        0          & -\d{25}{4}
    \end{pmatrix}
\]

Подставим значение в матрицу $A$:
\[
    A =
    \begin{pmatrix}
        \d{5}{2} & 1        \\
        0        & \d{5}{2}
    \end{pmatrix}
    \text{ - ЖНФ}
\]

\answer{4}
Линейный оператор $\varphi \colon \R^2 \to \R^2$, задаваемый матрицей:
\[
    A(\varphi, \e) =
    \begin{pmatrix}
        \d{5}{2} & 1        \\
        0        & \d{5}{2}
    \end{pmatrix}
\]

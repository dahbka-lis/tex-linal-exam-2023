\task{1}
\begin{condition}
    Определите все значения, которые может принимать размерность ядра линейного оператора $\varphi \colon \R^4 \to \R^4$ при условии, что в пересечении ядра и образа содержится вектор $v = (1, 2, 0, -1)$.
\end{condition}
С тем учётом, что в пересечении ядра и образа содержится вектор, смело можем утверждать, что размерность пересечения точно больше единицы:
\[
    \dim (\Ker \varphi \cap \Im \varphi) \geqslant 1
\]

Теперь оценим размерность ядра $\varphi$:
\[
    1 \leqslant \dim(\Ker \varphi \cap \Im \varphi) \leqslant \dim \Ker \varphi \leqslant \dim(\Ker \varphi + \Im \varphi) \leqslant 4 = \dim \R^4
\]

Отметим тот факт, что вектор $v$ лежит также в образе $\varphi$, с чего делаем вывод, что и размерность $\Im \varphi$ ненулевая. Тогда воспользуемся \textit{теоремой о связи размерностей ядра и образа линейного отображения}:
\begin{theorem}
    \[
        \dim \Im \varphi + \dim \Ker \varphi = \dim V
    \]
\end{theorem}

В нашем же случае $\dim \Ker \varphi + \dim \Im \varphi = \dim \R^4 = 4$, тогда понятно, что $\dim \Ker \varphi \in \{1, 2, 3\}$. Рассмотрим каждый случай и приведём примеры \textbf{(без примеров потеря баллов)}. Для каждого примера дополняем вектор $v$ векторами стандартного базиса $\R^4$, назовём базис $\e = (v, e_1, e_2, e_3)$:

\begin{enumerate}
    \item $\dim \Ker \varphi = 1$:{
          В таком случае можем построить отображение, которое переводит лишь вектор $v$ в нулевой. Матрица оператора будет иметь следующий вид в нашем базисе:
          \[
              A(\varphi, \e) =
              \begin{pmatrix}
                  0 & 1 & 0 & 0 \\
                  0 & 0 & 1 & 0 \\
                  0 & 0 & 0 & 1 \\
                  0 & 0 & 0 & 0
              \end{pmatrix}
              \implies
              \begin{cases}
                  v \to \vec{0} \\
                  e_1 \to v     \\
                  e_2 \to e_1   \\
                  e_3 \to e_2
              \end{cases}
          \]
          }

    \item $\dim \Ker \varphi = 2$:{
          В таком случае можем построить отображение, которое переводит векторы $v$ и $e_1$ в нулевой. Матрица оператора будет иметь следующий вид в нашем базисе:
          \[
              A(\varphi, \e) =
              \begin{pmatrix}
                  0 & 0 & 1 & 0 \\
                  0 & 0 & 0 & 0 \\
                  0 & 0 & 0 & 1 \\
                  0 & 0 & 0 & 0
              \end{pmatrix}
              \implies
              \begin{cases}
                  v, e_1 \to \vec{0} \\
                  e_2 \to v          \\
                  e_3 \to e_2
              \end{cases}
          \]
          }

    \item $\dim \Ker \varphi = 3$:{
          В таком случае можем построить отображение, которое переводит векторы $v$, $e_1$ и $e_2$ в нулевой. Матрица оператора будет иметь следующий вид в нашем базисе:
          \[
              A(\varphi, \e) =
              \begin{pmatrix}
                  0 & 0 & 0 & 1 \\
                  0 & 0 & 0 & 0 \\
                  0 & 0 & 0 & 0 \\
                  0 & 0 & 0 & 0
              \end{pmatrix}
              \implies
              \begin{cases}
                  v, e_1, e_2 \to \vec{0} \\
                  e_3 \to v
              \end{cases}
          \]
          }
\end{enumerate}

\answer{1}
Размерность ядра линейного оператора $\varphi$ может принимать значения от 1 до 3. Примеры приведены выше.

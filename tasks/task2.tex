\task{2}
\begin{condition}
    Приведите пример неопределённой квадратичной формы $Q \colon \R^3 \to \R$, принимающей отрицательные значения на всех ненулевых векторах подпространства:
    \[
        \{ (x, y, z) \in \R^3 \mid x - 2y + z = 0 \}
    \]
    Ответ представьте в стандартном виде многочлена 2-й степени от координат $x,y,z$.
\end{condition}

Пусть $S = \{ (x, y, z) \in \R^3 \mid x - 2y + z = 0 \}$. Тогда $S$ - подпространство решений ОСЛУ. Найдём базис через ФСР:
\[
    x - 2y + z = 0
    \implies
    \begin{amatrix}{3}{1}
        1 & -2 & 1 & 0
    \end{amatrix}
    \leadsto
    S =
    \langle
    \underbrace{
        \begin{pmatrix}
            -1 \\
            0  \\
            1
        \end{pmatrix}
    }_{f_1}, \;
    \underbrace{
        \begin{pmatrix}
            2 \\
            1 \\
            0
        \end{pmatrix}
    }_{f_2}
    \rangle
\]

Зададим квадратичную форму $Q$ так, чтобы $Q(S) < 0$. В таком случае векторы $f_1, f_2$ отобразим в отрицательные значения. Так как $Q$ задаём на $\R^3$ по условию, дополним базис $S$ до базиса всего пространства:
\[
    f_3 =
    \begin{pmatrix}
        0 \\
        0 \\
        1
    \end{pmatrix},
    \quad
    \begin{pmatrix}
        f_1 & f_2 & f_3
    \end{pmatrix}
    =
    \begin{pmatrix}
        -1 & 2 & 0 \\
        0  & 1 & 0 \\
        1  & 0 & 1
    \end{pmatrix}
    \leadsto
    \begin{pmatrix}
        1 & 0 & 0 \\
        0 & 1 & 0 \\
        0 & 0 & 1
    \end{pmatrix}
    \implies
    \f = (f_1, f_2, f_3) \text{ - базис в } \R^3
\]

\begin{theorem}
    $Q$ - неопределённая, если её индексы инерции $i_+, i_-$ равны некоторым ненулевым значениям.
\end{theorem}

Пусть значение квадратичной формы $Q$ отрицательно для $f_1, f_2$ (векторы базиса $S$), а для $f_3$ - положительно, тогда в базисе $\f$ она будет иметь нормальный вид с такой матрицей:
\[
    B(Q, \f) =
    \begin{pmatrix}
        -1 & 0  & 0 \\
        0  & -1 & 0 \\
        0  & 0  & 1
    \end{pmatrix},
    \quad
    \begin{cases}
        i_+ = 1 \\
        i_- = 2
    \end{cases}
\]

Мы задали неопределённую квадратичную форму в нормальном виде. Найдём её исходный вид, вспомнив \textit{Закон инерции}:

\begin{theorem}
    Числа $i_+$ и $i_-$ не зависят от выбора базиса, в котором $Q$ принимает нормальный вид.
\end{theorem}

В таком неопределённость нашей квадратичной формы $Q$ останется на месте при смене базиса.

Пусть $\e$ - базис, в котором $Q$ имеет стандартный вид, причём $\f = \e \cdot C$, где $C$ - матрица перехода. Полагаем, что $\e$ - стандартный базис, тогда соберём $C$ из векторов базиса $\f$:
\[
    C =
    \begin{pmatrix}
        -1 & 2 & 0 \\
        0  & 1 & 0 \\
        1  & 0 & 1
    \end{pmatrix}
\]

\newpage

Далее, для нахождения матрицы стандартного вида $Q$ рассмотрим формулу смены базиса квадратичной формы в терминах базисов $\f$ и $\e$:
\[
    B(Q, \f) = C^T \cdot B(Q, \e) \cdot C
    \;
    \implies
    \;
    B(Q, \e) = (C^{-1})^T \cdot B(Q, \f) \cdot C^{-1}
\]

Воспользуемся методом Гаусса для нахождения $C^{-1}$:
\[
    (C \mid E) =
    \begin{amatrix}{3}{3}
        -1 & 2 & 0 & 1 & 0 & 0 \\
        0 & 1 & 0 & 0 & 1 & 0 \\
        1 & 0 & 1 & 0 & 0 & 1
    \end{amatrix}
    \leadsto
    \begin{amatrix}{3}{3}
        1 & 0 & 0 & -1 & 2 & 0 \\
        0 & 1 & 0 & 0 & 1 & 0 \\
        0 & 0 & 1 & 1 & -2 & 1
    \end{amatrix}
    =
    (E \mid C^{-1})
\]

Найдем матрицу квадратичной формы $Q$ в базисе $\e$:
\[
    B(Q, \e) =
    \begin{pmatrix}
        -1 & 2  & 0 \\
        0  & 1  & 0 \\
        1  & -2 & 1
    \end{pmatrix}^T
    \begin{pmatrix}
        -1 & 0  & 0 \\
        0  & -1 & 0 \\
        0  & 0  & 1
    \end{pmatrix}
    \begin{pmatrix}
        -1 & 2  & 0 \\
        0  & 1  & 0 \\
        1  & -2 & 1
    \end{pmatrix}
    =
    \begin{pmatrix}
        0 & 0  & 1  \\
        0 & -1 & -2 \\
        1 & -2 & 1
    \end{pmatrix}
\]

Пусть $X = (x \; y \; z)$. Тогда по $B(Q, \e)$ выпишем стандартный вид квадратичной формы $Q$ от координат $x, y, z$:
\[
    Q(X) = 2xz - y^2 - 4yz + z^2
\]

\answer{2}
Пример подходящей квадратичной формы: $Q(X) = 2xz - y^2 - 4yz + z^2$.
